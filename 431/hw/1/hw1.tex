\documentclass{article}
\usepackage[left=1in,right=0.75in,top=1in,bottom=1in]{geometry}

\usepackage{amsmath}
\usepackage{amssymb}
\usepackage{graphicx}

\begin{document}

\textbf{Math 431 hw1}\\
\textbf{Harry LUO} \\
\begin{itemize}
      \item [\textbf{ex 1.2}]

      \item[a] Denote: Cereal as $C$, eggs as $E$, Fruits as $F$ .

            sample space: $\Omega = \{(C,E),(C,F), (E,F )\} $ \\

      \item[b] $A = \{(C,E),(C,F)\}$\\

      \item[\textbf{ex 1.6}]

      \item[a]For an urn with 3 $G$ and 4 $Y$, choose 2 balls randomly without replacement.\\
            Assuming that  the balls are only distinguishable by their color, and that the order matters, a possible sample space is
            $$\Omega = \{ (G,Y),(G,G),(Y,Y),(Y,G)\}$$\\
            Collection of events $A$ that contains two diffrent colored balls: $A = \{(G,Y),(Y,G)\}$ \\

      \item[b ]The total number of ways to choose 2 balls from 7 is \\$|\Omega|=\binom{7}{2} = 21$,\\
            To have 2 balls of different colors, we can draw 1 G and 1 Y. Thus, the number of ways to choose 2 balls of different colors is \\$|A|=\binom{3}{1} \binom{4}{1} = 12$.\\
            So, the probability of drawing 2 balls of different colors is $$\frac{|A|}{|\Omega|}=\frac{12}{21} = \frac{4}{7}$$\\


      \item[\textbf{ex 1.12}]
      \item[a]

            Probability that we need at most 3 rolls to see the number 4, while considering axiom 3:
            $$P(X\leq 3) = P(X=1)+P(X=2)+P(X=3)$$
            since it is a fair die, for each roll, we have a sample space:
            $$\Omega = \{1,2,3,4,5,6\}$$.
            For each roll, the collection of event would be one element in $\Omega$
            Thus,
            $$P(X=1) =\frac{1}{|\Omega|}= \frac{1}{6}$$
            If we roll 2 times until the expected number, it means that for the first roll, there is a $\frac{5}{6}$prabability of not having the desired num. And we have $\frac{1}{6}$ probability of having the desired number. Thus,
            $$P(X=2) = \frac{5}{6}\times\frac{1}{6}$$ Similarly, if we rolled 3 times, it would mean that we have $\frac{5}{6}\times \frac{5}{6}$chance of missing the number on the first 2 roll:
            $$P(X=3) = (\frac{5}{6})^2\times\frac{1}{6}$$ Therefore,

            $$P(X\leq 3) = \frac{1}{6}+\frac{5}{6}\times\frac{1}{6}+(\frac{5}{6})^2\times\frac{1}{6}=0.42$$\\

      \item[b] We conclude from the last section that, if it took us N times to roll a desired number using a fair die, it suggests that we have went through a $(\frac{5}{6})^{N-1}$ chance of rolling anything but the wanted number for N times.\\
            Therefore, the probability of rolling 2k times of die rolls is $$P(X=2k)=(\frac{5}{6})^{2k-1}\times \frac{1}{6}$$
            We thus write the probability for us needing an even number of rolls as the following:

            \begin{equation}
                  P(\text{even number of rolls})= \sum_{k=1}^{\infty}(\frac{5}{6})^{2k-1}\times \frac{1}{6} = \frac{\frac{5}{6}\times \frac{1}{6}}{1-(\frac{5}{6})^2}=\frac{5}{11}
            \end{equation}

      \item[\textbf{ex 1.22}]
      \item[a] A stack of cards contains 4 groups(hearts, diamonds, clubs, spades) of 13 cards each. We denote J as 11, Q as 12, K as 13, A as 14; denote heards as H, diamonds as D, clubs as C, and Spades as S. When we draw 1 card uniformly at random, the sample space could be written as:
            \begin{equation}
                  \Omega = \{(N_i,G_i)|N_i\in\{2,3,\dots,14\},G_i\in\{H,D,C,S\} \}
            \end{equation}
            Since we uniformly draw a card, the probability of picking any spcific card would be $P(Draw)\frac{1}{52}$.

      \item[b]since the probability of drawing a specific card is $\frac{1}{52}$, $P(A)=\frac{3}{53}=\frac{1}{53}+\frac{1}{53}+\frac{1}{53}$ could be seen as the event of picking 3 specific cards at a time.

      \item[c] In a probability space where we are dealing with a standard deck of 52 cards, the probabilities of any event are determined by the ratios of the number of favorable outcomes to the total number of outcomes (52).
            To show that no event has a probability of $\frac{1}{5}$, we need to show that there is no integer $n$ such that $\frac{n}{52}=\frac{1}{5}$.  Since 52 is not divisible by 5, there is no such integer $n$ and thus no event has a probability of $\frac{1}{5}$.

      \item[\textbf{ex 1.28}]
      \item[a]when sampled without replacement, and considering 2 balls are drawn, The sample space \( \Omega \) for the experiment of drawing two balls without replacement from an urn with \( m \) green balls and \( n \) yellow balls can be described as follows:
            \begin{equation*}
                  \Omega = \{ (G_i, G_j), (Y_k, Y_l), (G_i, Y_k), (Y_k, G_i) \mid 1 \leq i < j \leq m, 1 \leq k < l \leq n \}
            \end{equation*}
            where \( G \) denotes a green ball, \( Y \) denotes a yellow ball, and the subscripts \( i, j, k, l \) represent individual balls.

            The collection of events \( F \), which is a set of subsets of \( \Omega \), includes, among others, the following events of interest:
            \begin{align*}
                  E_G             & = \{ (G_i, G_j) \mid 1 \leq i < j \leq m \} \text{, the event of drawing two green balls,}  \\
                  E_Y             & = \{ (Y_k, Y_l) \mid 1 \leq k < l \leq n \} \text{, the event of drawing two yellow balls,} \\
                  E_{\text{same}} & = E_G \cup E_Y \text{, the event of drawing two balls of the same color.}
            \end{align*}
            The probability of drawing 2 balls of the same color is
            \begin{equation*}
                  P(E_{\text{same}}) = \frac{|E_{\text{same}}|}{|\Omega|} = \frac{\binom{m}{2} + \binom{n}{2}}{\binom{m+n}{2}} = \frac{m(m-1) + n(n-1)}{(m+n)(m+n-1)}
            \end{equation*}

      \item[b] When sampled with replacement, the sample space would now be simply has the size of $(m+n)^2$, since we draw twice from an urn with $(m+n)$ balls.
            Following the same reasoning, the probability of drawing two balls of the same color is
            \begin{equation*}
                  P(E_{\text{same}}) = \frac{|E_{\text{same}}|}{|\Omega|} = \frac{m^2 + n^2}{(m+n)^2}
            \end{equation*}

      \item[\textbf{ex 1.32}]
            The sample space $\Omega$ is the set of all possible outcomes, which can be represented as:
            \[
                  \Omega = \left\{ (Rank_i, Suit_j) \mid Rank_i \in \{2, 3, \ldots, 10, J, Q, K, A\}, Suit_j \in \{\text{Hearts}, \text{Diamonds}, \text{Clubs}, \text{Spades}\} \right\}
            \]
            A full house, as given in the question, is a union of two events: the event of drawing three cards of the same rank and the event of drawing two cards of another same rank. The probability of drawing a full house is then the sum of the probabilities of these two events.
            There are 13 different ranks and we want to choose 1 of them for the three-of-a-kind. With the chosen rank, there are $\binom{4}{3}$ ways to choose the three cards of the same rank. $$P(\rm \text{3 of a kind})= 13\times \binom{4}{3}$$
            Then, there are 12 remaining ranks and we want to choose 1 of them for the two-of-a-kind. With the chosen rank, there are $\binom{4}{2}$ ways to choose the two cards of the same rank. $$P(\rm \text{2 of a kind})= 12\times \binom{4}{2}$$
            The total number of ways to choose 5 cards from 52 is $\binom{52}{5}$.
            Thus, the probability of drawing a full house is \begin{equation*}
                  P(\text{Full House}) = \frac{13\times \binom{4}{3} \times 12\times \binom{4}{2}}{\binom{52}{5}}
            \end{equation*}


      \item[ex 1.34]  The circle's centea must be at least $\frac{1}{3}$ away from each side of the square. Thus, if we put the lower-left corner of the square at the origin of a cartesian coordinate,
            then the sample space $\Omega$ is defined as the set of all points inside the unit square:
            \[ \Omega = \{(x, y) \mid 0 < x < 1, 0 < y < 1\} \]

            The event $E$ that the circle lies entirely inside the square is given by:
            \[ E = \{(x, y) \mid \frac{1}{3} < x < \frac{2}{3}, \frac{1}{3} < y < \frac{2}{3}\} \]

            The probability $P$ of the circle being entirely inside the square is the ratio of the area of $E$ to the area of $\Omega$:
            \[ P(E) = \frac{\text{Area of } E}{\text{Area of } \Omega} = \frac{\left(\frac{1}{3}\right)^2}{1^2} = \frac{1}{9} \]

      \item[\textbf{ex 1.36}]
      \item[a]  Given a uniform random point $(X, Y)$ inside the unit square $[0, 1] \times [0, 1]$, since the distribution is uniform, the probability that the x-coordinate lies in a specific interval $(a, b)$ is given by the length of the interval:
      \[ P(a < X < b) = b - a \]
      
      (b) The probability that the absolute difference between the x-coordinate and y-coordinate is less than or equal to $\frac{1}{4}$ can be visualized as two bands on either side of the line $Y = X$ within the unit square. Each band has a width of $\frac{1}{4}$, thus the total area of these bands represents the probability:
      \[ P(|X - Y| \leq \frac{1}{4}) = \text{Area of the bands} = \frac{3}{8}\]
      
      









\end{itemize}


\end{document}